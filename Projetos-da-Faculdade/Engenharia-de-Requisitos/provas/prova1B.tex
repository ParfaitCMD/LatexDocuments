\documentclass{article}
\usepackage[utf8]{inputenc}
\usepackage[T1]{fontenc}
\usepackage{fourier} % Pacote que carrega a fonte Utopia
\usepackage{hyperref}

\begin{document}

\section*{Qualidade de Software}

A qualidade de software é dividida em três atividades principais:

\begin{itemize}
    \item \textbf{Garantia da Qualidade:} Esta atividade estabelece procedimentos e padrões de desenvolvimento para assegurar a criação de um software de qualidade.

    \item \textbf{Planejamento da Qualidade:} Trata-se do processo de criação de um plano de qualidade detalhado, que será aplicado a um processo específico de desenvolvimento.

    \item \textbf{Controle da Qualidade:} Esta atividade tem como objetivo garantir que o processo de desenvolvimento especificado no planejamento seja seguido, verificando a adesão aos padrões e procedimentos estabelecidos.
\end{itemize}

---

\subsection*{Níveis de Preocupação com a Qualidade}

Segundo Sommerville (2011), existem três níveis de preocupação para que a qualidade de um produto de software seja alcançada.

\begin{itemize}
    \item \textbf{Preocupação no nível organizacional:} o gerenciamento se preocupa em estabelecer um quadro de processos organizacionais e padrões que resultarão em um software de alta qualidade.

    \item \textbf{Preocupação no nível do projeto 1:} o gerenciamento envolve a aplicação de processos de qualidade específicos e a verificação sobre o seu seguimento, garantindo que tudo esteja correto.

    \item \textbf{Preocupação no nível do projeto 2:} o gerenciamento da qualidade preocupa-se com o desenvolvimento de um plano de qualidade para o projeto. Nesse plano, devem estar descritas as metas de qualidade e a definição de processos e padrões que serão utilizados.
\end{itemize}

\end{document}